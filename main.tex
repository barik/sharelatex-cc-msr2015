\documentclass[conference]{IEEEtran}
\usepackage{IEEEpreamble}

\usepackage{booktabs}
\usepackage{blindtext}

\begin{document}
%
% paper title
% can use linebreaks \\ within to get better formatting as desired
\title{\textsc{Fuse}: A Reproducable, Internet-scale Dataset\\of Publicly-indexed Spreadsheets for Science}

\author{
\IEEEauthorblockN{Titus Barik\IEEEauthorrefmark{1}\IEEEauthorrefmark{2}}
\IEEEauthorblockA{\IEEEauthorrefmark{1}ABB Corporation Research\\
Raleigh, North Carolina, USA\\
titus.barik@us.abb.com
}
\and
\IEEEauthorblockN{
Kevin Lubick,
Justin Smith,\\
John Slankas,
Emerson Murphy-Hill}
\IEEEauthorblockA{
\IEEEauthorrefmark{2}North Carolina State University, Raleigh, USA\\
\{kjlubick, jssmit11\}@ncsu.edu, emerson@csc.ncsu.edu}
\and
\IEEEauthorblockN{Felienne Hermans}
\IEEEauthorblockA{Delft University of Technology\\
Delft, Netherlands\\
f.f.j.hermans@tudelft.n
}
}

% use for special paper notices
%\IEEEspecialpapernotice{(Invited Paper)}

% http://www.cra.org/resources/bp-view/evaluating_computer_scientists_and_engineers_for_promotion_and_tenure/

% make the title area
\maketitle


\begin{abstract}
We submit a corpus consisting of 1.5 million spreadsheets, extracted using the Common Crawl corpus (based on Blekko). This corpus is compared against a proprietary index from a leading search engine company to measure representative against other commercial index. This corpus is intended to replace EUSES.
\end{abstract}
% IEEEtran.cls defaults to using nonbold math in the Abstract.
% This preserves the distinction between vectors and scalars. However,
% if the conference you are submitting to favors bold math in the abstract,
% then you can use LaTeX's standard command \boldmath at the very start
% of the abstract to achieve this. Many IEEE journals/conferences frown on
% math in the abstract anyway.

% no keywords




% For peer review papers, you can put extra information on the cover
% page as needed:
% \ifCLASSOPTIONpeerreview
% \begin{center} \bfseries EDICS Category: 3-BBND \end{center}
% \fi
%
% For peerreview papers, this IEEEtran command inserts a page break and
% creates the second title. It will be ignored for other modes.
\IEEEpeerreviewmaketitle

\section{Call for Papers}

\textbf{Data papers}. We want to encourage researchers to share their data. Data papers should describe data sets curated by their authors and made available to others. They are expected to be at most 4 pages long and should address the following: description of the data, including its source; methodology used to gather it; description of the schema used to store it, and any limitations and/or challenges of this data set. The data should be made available at the time of submission of the paper for review, but will be considered confidential until publication of the paper. Further details about data papers are available on the conference website. 

\begin{enumerate}
\item Description of the data.
\item Methodology used to gather it
\item Description of the schema used to store it
\item Limitations and Challenges of data set
\end{enumerate}

\section{Introduction}

spreadsheets as end-user programmers; focus on how spreadsheets are useful for software engineerig


all the problems
euses sucks~~\cite{Fisher2005}

similar in size to other large corpora
representativeness against Google index
metadata

helps with tool evaluation

accessible -- getting this for yourself is a high cost maybe 5k

this set is reproducable
compared with existing data sets, it is new -- the others are more than a decade old.

even the enron corpus is 15000 spreadsheets -- whoopee. But it's good to compare against since it's a private corpus.

The contribution of this paper is:

\begin{itemize}
\item Foo
\end{itemize}

go to figshare
% http://figshare.com/articles/Enron_s_Spreadsheets_and_Related_Emails_A_Dataset_and_Analysis/1222882

\subsection{Subsection Heading Here}
Subsection text here.

need a fancy table

what metrics?

happy medium between euses and google

\subsubsection{Subsubsection Heading Here}
Subsubsection text here.

deliberate design decision to 
intentionally spluit the data into metadata and binary phases


% An example of a floating figure using the graphicx package.
% Note that \label must occur AFTER (or within) \caption.
% For figures, \caption should occur after the \includegraphics.
% Note that IEEEtran v1.7 and later has special internal code that
% is designed to preserve the operation of \label within \caption
% even when the captionsoff option is in effect. However, because
% of issues like this, it may be the safest practice to put all your
% \label just after \caption rather than within \caption{}.
%
% Reminder: the "draftcls" or "draftclsnofoot", not "draft", class
% option should be used if it is desired that the figures are to be
% displayed while in draft mode.
%
%\begin{figure}[!t]
%\centering
%\includegraphics[width=2.5in]{myfigure}
% where an .eps filename suffix will be assumed under latex,
% and a .pdf suffix will be assumed for pdflatex; or what has been declared
% via \DeclareGraphicsExtensions.
%\caption{Simulation Results}
%\label{fig_sim}
%\end{figure}

% Note that IEEE typically puts floats only at the top, even when this
% results in a large percentage of a column being occupied by floats.


% An example of a double column floating figure using two subfigures.
% (The subfig.sty package must be loaded for this to work.)
% The subfigure \label commands are set within each subfloat command, the
% \label for the overall figure must come after \caption.
% \hfil must be used as a separator to get equal spacing.
% The subfigure.sty package works much the same way, except \subfigure is
% used instead of \subfloat.
%
%\begin{figure*}[!t]
%\centerline{\subfloat[Case I]\includegraphics[width=2.5in]{subfigcase1}%
%\label{fig_first_case}}
%\hfil
%\subfloat[Case II]{\includegraphics[width=2.5in]{subfigcase2}%
%\label{fig_second_case}}}
%\caption{Simulation results}
%\label{fig_sim}
%\end{figure*}
%
% Note that often IEEE papers with subfigures do not employ subfigure
% captions (using the optional argument to \subfloat), but instead will
% reference/describe all of them (a), (b), etc., within the main caption.


% An example of a floating table. Note that, for IEEE style tables, the
% \caption command should come BEFORE the table. Table text will default to
% \footnotesize as IEEE normally uses this smaller font for tables.
% The \label must come after \caption as always.
%
%\begin{table}[!t]
%% increase table row spacing, adjust to taste
%\renewcommand{\arraystretch}{1.3}
% if using array.sty, it might be a good idea to tweak the value of
% \extrarowheight as needed to properly center the text within the cells
%\caption{An Example of a Table}
%\label{table_example}
%\centering
%% Some packages, such as MDW tools, offer better commands for making tables
%% than the plain LaTeX2e tabular which is used here.
%\begin{tabular}{|c||c|}
%\hline
%One & Two\\
%\hline
%Three & Four\\
%\hline
%\end{tabular}
%\end{table}


% Note that IEEE does not put floats in the very first column - or typically
% anywhere on the first page for that matter. Also, in-text middle ("here")
% positioning is not used. Most IEEE journals/conferences use top floats
% exclusively. Note that, LaTeX2e, unlike IEEE journals/conferences, places
% footnotes above bottom floats. This can be corrected via the \fnbelowfloat
% command of the stfloats package.

\section{Data Extraction}

needed to create WAT index files

% dremel> select COUNT(doc.DocId) FROM docjoin.base WHERE doc.content.ContentType = 23;
% +------------------+
% | COUNT(doc.DocId) |
% +------------------+
% |          1635142 |
% +------------------+
% WARNING: Partition skipped
% WARNING: ~0.0% of data was not scanned (see "settings min_completion_ratio")
% 1 row in result set (319.28 sec)
% Scan rate: 37.38M rows/sec, SWE cost: 7.28138s

% dremel> select doc.DocId, RIGHT(doc.URL, 30), doc.Pagerank FROM docjoin.base WHERE doc.content.ContentType = 23 L
% IMIT 20;
% +----------------------+--------------------------------+--------------+
% | doc.DocId            | RIGHT(doc.URL, 30)             | doc.Pagerank |
% +----------------------+--------------------------------+--------------+
% | 18262361551973990081 | Noticias/210111pssufal2011.xls |        45696 |
% | 18262514318872388335 | E7%99%BB%E8%AE%B0%E8%A1%A8.xls |        53441 |
% | 18262628117481929248 | ameck-cd57ffgym.fr/file/31037/ |        49536 |
% | 18262462250520031836 | uation%20risque%20chimique.xls |            1 |
% | 18262515209375261799 | 9206/file/4%20LISTE%202014.xls |        51664 |
% | 18262601711327171729 | d_org=200054&id_dokumenty=1069 |        51318 |
% | 18262429851281169486 | loads/2009/11/distributori.xls |        47265 |
% | 18262677137761586621 | /03_Zahlungen_Schulversion.xls |        49376 |
% | 18262497829673811111 | FULL_SCHOOL_LIST_1982-1983.xls |        55771 |
% | 18098709049136093897 | 1_Guelleberechnungstabelle.XLS |        48219 |
% | 18098336056523698774 | 46658c58776252a270492189fd.xls |        51278 |
% | 18098420930462689378 | 2nd_and_3rd_Posting)_FINAL.xls |        46743 |
% | 18221693724751189184 | /stories/papka_bossa/rosta.xls |            1 |
% | 18221288433906859741 | mediaprodukciyu-22.12.2013.xls |        45065 |
% | 18221490419625251728 | 75817255013/files/sadowara.xls |        52813 |
% | 18221566511153360853 | ostnica-arhitektura_plosca.xls |        49301 |
% | 18221440248932413652 | 83%D1%81%D0%BA%D0%B8%D0%B9.xls |        44383 |
% | 18221602484570358598 | ents/0000000/44/25ichiran5.xls |        52927 |
% | 18221337705568163749 | es/201011/2010111710570042.xls |        52262 |
% | 18221462757268787142 | 20%D0%AF%D0%9D%D0%90%D0%9E.xls |        49763 |
% +----------------------+--------------------------------+--------------+
% 20 rows in result set (26.13 sec)
% Scan rate: 0.04M rows/sec, SWE cost: 6.4e-05s

how did we decide to filter? MSDN content-type

% http://blogs.msdn.com/b/vsofficedeveloper/archive/2008/05/08/office-2007-open-xml-mime-types.aspx

google obtained from proprietary database query that extracted from google index all spreadsheets with content type application/ms-excel.

common crawl
segments -> subproblems

docjoiner

\blindtext[3]
\blindtext[3]

\emph{cleaning}

\section{Description of Spreadsheet Corpus}

\begin{table*}[!t]
%% increase table row spacing, adjust to taste
%\renewcommand{\arraystretch}{1.3}
% if using array.sty, it might be a good idea to tweak the value of
% \extrarowheight as needed to properly center the text within the cells
\caption{A comparison of the EUSES, Common Crawl Index, and Proprietary Index\label{table_example}}
\centering
%% Some packages, such as MDW tools, offer better commands for making tables
%% than the plain LaTeX2e tabular which is used here.
\begin{tabular}{llll}
\toprule
& \textbf{EUSES} & \textbf{Common} & \textbf{Prop. Search Index}\\
\midrule
$n$ & 6,000 & 600,000 & $>$ 1,000,000\\
size & 23.3 & 32 & 2\\
Five & Six\\
\bottomrule
\end{tabular}
\end{table*}



A summary of your spreadsheet corpus can be found in Table~\ref{table_example}.

\section{Data Schema}

how do you get it?

csv file
mongodb recordobject warc extracts


% TODO(tbarik): Switch this table.


\section{Dataset Limitations}



\section{Conclusion}
The conclusion goes here.

\section{Related Work}

why do people use spreadsheets?

what other corpora exists?


\textbf{Existing corpora.}
\textbf{Spreadsheet tools.}



% conference papers do not normally have an appendix


% use section* for acknowledgement
\section*{Acknowledgment}

I did this all by myself.





% trigger a \newpage just before the given reference
% number - used to balance the columns on the last page
% adjust value as needed - may need to be readjusted if
% the document is modified later
%\IEEEtriggeratref{8}
% The "triggered" command can be changed if desired:
%\IEEEtriggercmd{\enlargethispage{-5in}}

% references section

\raggedright
% can use a bibliography generated by BibTeX as a .bbl file
% BibTeX documentation can be easily obtained at:
% http://www.ctan.org/tex-archive/biblio/bibtex/contrib/doc/
% The IEEEtran BibTeX style support page is at:
% http://www.michaelshell.org/tex/ieeetran/bibtex/
\bibliographystyle{IEEEtran}
% argument is your BibTeX string definitions and bibliography database(s)
\bibliography{library}


% that's all folks
\end{document}


